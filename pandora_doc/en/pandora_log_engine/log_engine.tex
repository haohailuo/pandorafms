\documentclass[a4paper,10pt]{article}


%opening
\title{Kind of guide to the Pandora FMS's log agent}
\author{Jose Navarro}

\begin{document}

\maketitle

\tableofcontents

\pagebreak

\begin{abstract}
The Pandora 1.2 agent's code have been improved in order to increase its capabilities for monitoring and analyzing text files that grow and are rotated, like typically log files do. Perl regular expressions and code can be used in the design of modules. From this aproximation, every new line of the monitored text is considered an individual piece of information. Another secondary improvements have been done to Pandora FMS, like the use of regular expressions to define alerts on data string modules.
\end{abstract}

\section{Notes}

This chapter is documentation for developers of the Pandora project, and its main objective is to share information about the inners workings of the code. It also can be used as a installation and configuration guide till the official documentation is attached to the trunk branch of the project.
\section{Introduction}

      Till version 1.2, pandora was able to treat several types
      of data, including numeric, incremetal numeric and strings. However,
      string data type lacks some funcionalities for monitoring log files.

      The aim of this pack of code, or development branch, is to strength
      the capabilities of Pandora for monitoring text files that are
      increased and rotated with the time, like log files.

      In particular, the main objectives of this development branch are:

\begin{itemize}
\item 	    pandora agents should be able to analize new lines of a
	    text log file, and only the new ones
\item 	    pandora agents should treat each new line as a independent
	    element/data unit
\item 	    pandora agents need to be aware about rotation of log files
	    in order to not lose information
\item   alerts capabilities on string data types has to be improved
\item   graphic representation capabilites on string data modules has
	    to be improved
\end{itemize}

      For a technical summary of new technical features, please refer to section \ref{sec:techsummary} on the page \pageref{sec:techsummary}.

\section{Architecture}

      Funcionalities described in the Introduction chapter can be implemented
      in several ways. The approach presented in this document propose that
      the pandora agent can control log files, detect which new lines are
      created since its last execution, and process differents modules on
      EACH of the lines.

      On this branch of development, pandora\_agent.sh calls to another script
      just before copying data files to the server. This script is a simple
      perl script named pandora\_agent\_log.pl, that uses another file as 
      a configuration file, pandora\_agent\_log.conf.

      pandora\_agent\_log.pl and pandora\_agent\_log.conf are very similar to
      pandora\_agent.sh and pandora\_agent.conf. In the future, its 
      funcionalities are intended to be included in the agent code. Now,
      they are presented sepparated just for clarity in the development.

In order to monitor only the new lines that are added to a text file between two executions of the agent, index files are used. This files contain a pointer to the last byte read in the last execution. If the index file does not exist, Pandora will create it, but \emph{will not analyze lines in this first interaction}. This avoid overloading of the agent the first time that tries to analyze a log file with, maybe, thousands of lines. See the section of the index file for more information and more uses of this file.


\section{Summary of technical features}\label{sec:techsummary}

\subsection{line by line text file monitoring}

The log agent can monitor every new line of a text file, executing all the modules associated. Every line of the log generates a XML data file that is introduced in the Pandora FMS system with the information of all the modules executed.

For more information, refer to the commands  module\_log, module\_exec  in the section \ref{sec:agent.conf}.

\subsection{rotating log file monitoring}

The log agent detects when a log file has been rotated since de last execution, analyzing the new lines of the log file but also the lines not analyzed in the rotated file. This process is transparent to Pandora.

For more information, refer to the commands  module\_log\_rotated  in the section \ref{sec:agent.conf}.

\subsection{alerts on generic\_data\_string}

Pandora FMS can now use alerts on generic\_data\_string modules, firing the alert when a perl regular expression is matched.

For more information, refer to the section \ref{sec:alerts}.

\subsection{storing even repeated module data}

The default behaviour of Pandora is not to store repeated and consecutive data of a module in the database. This issue could be a problem when managing log files. A new feature has been development in order to store all the values returned by a module, even the repeated and consecutive ones. Note that this feature can apply to \emph{any} type of module.

For more information, refer to the commands  module\_store\_all\_data  in the section \ref{sec:agent.conf}.

\subsection{timestamp forging}

By default, the timestamp of a data XML file is determined by the time in which the agent has been executed. A new feature has been added that allows to forge the timestamp of the XML file. Note that this feature can apply to \emph{any} type of module.

For more information, refer to the commands  module\_log\_timestamp  in the section \ref{sec:agent.conf}.

\subsection{counters}

The log agent analyze only the group of new lines appended to a log file between executions, but it analyze one by one. Counters allow to count the number of these lines that match a regular expressions. 

For more information, refer to the commands  module\_log\_counter  in the section \ref{sec:agent.conf}. Refer also to the Index file section \ref{sec:indexfile}



\section{pandora\_agent\_log.pl}
    
      Some of the code used for this script, mainly the index files manegement,
      is based in the part of the code of pandora\_server/bin/
      pandora\_snmpconsole.pl
    
    
      this script performs the following actions in the following order:
    
\begin{itemize}
        
          
\item	    load the pandora\_agent\_log.conf configuration file, that is
	    described later in this chapter
	  
        
        
          
\item	    for each log file to be monitored loops through the next steps
	  
        
        
          
\item	    loads or creates an index file. Each index file stores 
	    information regarding the state of the log file in the last 
	    execution of the agent. Indexes are stored in \$PANDORA\_HOME. If this file does not exists, it is created pointing to the end of the log file. Nothing more is done for this log file.
	  
        
        
          
\item	    the script make some checks over the log file to see if it has
	    been rewritten and/or rotated. If it has been rotated, rotated
	    log file is processed first to recover last lines.
	  
        
        
          
\item	    loops for the NEW lines of the log (since last agent's 
	    execution). For each line, a data file in data\_out is created.
	  
        
        
          
\item	    each module associated with that log file is executed against
	    the new log line, and data is written in the corresponding data
	    file
	  
        
        
          
\item	    checksum is performed on data files
	  

\item 	index file is updated

        
\end{itemize}





\section{pandora\_agent\_log.conf}\label{sec:agent.conf}

      the structure of this file is the same as the structure of 
      pandora\_agent.conf, but some extensions has been added.

\begin{itemize}
\item        module\_log [LOG FILE] :  only the modules with `module\_log' are 
	considered. [LOG FILE] is the file, path included, of the log to
	be analysed. Different modules can be associated to the same log
	file. A module can only be associated (for now) to a log file.
      
\item        module\_log\_timestamp  :  timestamp of the data file can be rewritten
	using this module. The overriding timestamp is the result of 
	processing 'module\_exec' on the log line. Modules with 
	module\_log\_timestamp are not further considered as pandora modules.
	So, they require no name, description, data type, ...
     
\item      	module\_log\_rotated [LOG FILE] :  tells the agent what wil be the name
      	of the rotated log file. Everytime that the agent detects a rotation
      	in the main log file, it will analize the last lines of [LOG FILE]. 
      	You don't need to put it in every module associated with a log file:
      	in one is enough.
      
\item        module\_exec [EXPR]  :  expression to be executed on every new line for
	this module. For now, EXPR is a perl expression. You can use the 
	variable \$LINE to represent the new log line. F.ex., \\
	`\verb|module_exec $LINE =~ y/A-Z/a-z/; return $LINE;|' lowercase the log
	lines before be stored in pandora database.
     
\item        module\_store\_all\_data  :   the default behaviour of pandora is not to
        store in the database repeated values captures by the agent. This 
        options override that behaviour, forcing pandora to store ALL data.
        NOTE:  please note that this parameter can be used with all kind of
        modules, not just module\_log ones.

\item 	module\_log\_counter   :   makes the current module a counter. A counter is a special module that: (a) does neather create a module in the database nor in the web console (b) counts how many lines from the current execution of the agent matches the expression in \verb|module_exec|. For example, if the monitored log file /tmp/jarl.log has 5 new lines, this modules will return a number from 0 to 5 depending on the number of lines that match \verb|module_exec|. The results are stored in the index file of the corresponding log. See the index file section for more information.

\end{itemize}


\section{Index file}\label{sec:indexfile}

This section describes the functions and syntaxis of the index files used for the log agent.

\subsection{objectives}

Index files has the following main objectives:

\begin{itemize}
\item to store a pointer to the last line monitored in the last execution of the agent. Every time the agent executes, looks for this pointer and only analizes the new lines added.
\item to store information that cannot be introduced directly in Pandora FMS, like counters. This information requires another agent to be treated and introduced in the system.
\end{itemize}

One index file \emph{should} exist for every text file monitored. If this index file does not exist, the log agent will create one, \emph{but will not analyze the existing lines}. This means that the default behaviour of the log agent the first time it is executed is:

\begin{itemize}
\item to jump all the lines of the monitored log without analyzing them,  and 
\item to write a index file that makes that the agent will monitor \emph{only the new lines} the next time it is executed
\end{itemize}

This behaviour is very useful the first time Pandora is executed to monitor existing huge log files, avoiding overloading the first time the log agent tries to analyze (maybe) thousands of lines.

\subsection{syntax}

The index file is a text file that is stored in the same folder as the agent script. Its name is the complete path to the log file, where the `/'  and `$\backslash$' are substituted for `\_', and a `.index' extension is appended. For example, the index file corresponding to the text file `/home/user/teletubi.txt' should be `\_home\_user\_teletubi.txt.index'.

The contents of this index file are organized in lines:

\begin{enumerate}
\item \verb|last-byte-read 0 0|    the last two zeros are not used already
\item \verb|counter value-of-the-counter module-name|   where the first `counter' is literally this string, the value of the counter is the number of lines of the last agent execution matches module\_exec of the module \verb|module-name|.  There is a line with this format for every counter present in the configuration file that has a value of counter different than zero.
\item \verb|# log-file-name|   the character `\#' and the complete path to the log file
\end{enumerate}

\emph{NOTE:}   please, note that the lines corresponding to the counters only appear when the value of the counter is greater than zero, and that are actualized with every execution of the log agent. If you want to use this information to build a pandora module, place this module in the agent log or in an agent that executes with the same frequency than it.



\section{alert configuration on generic\_data\_string modules}\label{sec:alerts}
  
  It is possible to configure alerts on
  generic\_data\_string modules using perl regular expressions.
  
  
  NOTE:  please note that, although this feature has been development in this
  development branch, it can be used in all generic\_data\_string modules, not
  only in the module\_log ones.
  
  
  For the configuration of this feature, a new field has been added to the 
  `Alert association form' of the pandora console: `Perl expression'. A 
  succesful matching of the generic\_data\_string data will trigger an alert.
  
  
  Regular expressions has to have Perl syntax, for example:
  
\begin{itemize}
    
\item      \verb|word| : matches the word `word'
    
    
\item      \verb|^#| : matches lines beginning with `\#'

    
\item      \verb|(d{1,3}\.){3}\d{1,3}| :  matches IP addresses
    
\end{itemize}



\section{Example}\label{sec:example}

\subsection{objectives}

    Next pandora\_agent\_log.conf performs the following actions:

\begin{itemize}      
        
\item      	monitors log file /tmp/log1.log with two modules. First one, returns 
      	the line, and second one makes a simple substitution.
      	
      
\item      	all log lines of /tmp/log1.log are stored and displayed in database,
      	even repeated and consecutives ones.
      	for /tmp/log1.log, a rotated file is configured, /tmp/log1.log.0
      	
      
\item      	/tmp/log2.log log file is also monitored. A generic\_data module is 
      	configured.
      	
        
\item      	for /tmp/log2.log, a fixed timestamp is forced, so all data is 
      	registered in database as captured `2006/9/25 1:1:1'

\item 		for /tmp/log2.log, create a counter that counts how many lines contain the letter \verb|a|, and a module named `modcountlog2' that introduces this information in Pandora.
    
\end{itemize}

\subsection{pandora\_agent\_log.conf}

\begin{verbatim}

    
module_begin
module_name log1
module_descripcion log 1 log file
module_log /tmp/log1.log
module_log_rotated /tmp/log1.log.0
module_store_all_data
module_type generic_data_string
module_exec return $LINE
module_end

module_begin
module_name log1_subst
module_descripcion simple substitution in log 1
module_log /tmp/log1.log
module_type generic_data_string
module_exec $LINE =~ s/o/X/g; return $LINE
module_end

module_begin
module_name log2
module_description log2 - generic_data
module_log /tmp/log2.log
module_type generic_data
module_exec return $LINE;
module_end

module_begin
module_log /tmp/log2.log
module_log_timestamp
module_exec return '2006/9/25 1:1:1';
module_end

module_begin
module_log /tmp/log2.log
module_log_counter
module_name countlog2
module_description counter on log2. This is not considered a module for Pandora.
module_exec return ($LINE=~/a/)?1:0
module_end

module_begin
module_name modcountlog2
module_description density of lines containing the letter a
module_type generic_data
module_exec grep countlog2 _tmp_log2.log.index | cut -d' ' -f2
module_end


\end{verbatim}


\section{Installation}

Till the merging of this development branch to the trunk code of Pandora FMS, some tricks and tips are needed for the installation process. This installation process consist on this steps:

\subsection{Installing Pandora FMS 1.2}

The code of the svn is based in Pandora FMS 1.2, so begin installing this. The svn branch for the log agent does not contain the entire system due for performance reasons and, let's say it, inexperience and lazyness of myself.

\subsection{Substitute some branches}

The relevant branches of the log agent (modified) are:

\begin{itemize}
\item pandora\_console
\item pandora\_agents/linux
\item pandora\_server
\end{itemize}

\subsection{Follow the normal installation process}

ditto

\subsection{Notes on database structure}

For the agent log to work properly, two new fields have been added to the database (with respect to Pandora FMS 1.2). For your information, these are the two tables affected:

field `perl\_expr' in talerta\_agente\_modulo
\begin{verbatim}
# Table: 'talerta_agente_modulo'
# 

CREATE TABLE `talerta_agente_modulo` (
  `id_aam` int(11) unsigned NOT NULL auto_increment,
  `id_agente_modulo` int(11) NOT NULL default '0',
  `id_alerta` int(11) NOT NULL default '0',
  `al_campo1` varchar(255) default '',
  `al_campo2` varchar(255) default '',
  `al_campo3` mediumtext default '',
  `descripcion` varchar(255) default '',
  `dis_max` bigint(12) default NULL,
  `dis_min` bigint(12) default NULL,
  `time_threshold` int(11) NOT NULL default '0',
  `last_fired` datetime NOT NULL default '0000-00-00 00:00:00',
  `max_alerts` int(4) NOT NULL default '1',
  `times_fired` int(11) NOT NULL default '0',
  `module_type` int(11) NOT NULL default '0',
  `min_alerts` int(4) NOT NULL default '0',
  `internal_counter` int(4) default '0',
  `perl_expr` text default NULL,
  PRIMARY KEY  (`id_aam`)
) TYPE=InnoDB;
\end{verbatim}

field `store\_all\_data' in tagente\_modulo
\begin{verbatim}
# Database: pandora
# Table: 'tagente_modulo'
# 
CREATE TABLE `tagente_modulo` (
  `id_agente_modulo` bigint(100) unsigned NOT NULL auto_increment,
  `id_agente` int(11) NOT NULL default '0',
  `id_tipo_modulo` int(11) NOT NULL default '0',
  `descripcion` varchar(100) NOT NULL default '',
  `nombre` varchar(100) NOT NULL default '',
  `max` bigint(20) default '0',
  `min` bigint(20) default '0',
  `module_interval` int(4) unsigned default '0',
  `tcp_port` int(4) unsigned default '0',
  `tcp_send` varchar(150) default '',
  `tcp_rcv` varchar(100) default '',
  `snmp_community` varchar(100) default '',
  `snmp_oid` varchar(255) default '0',
  `ip_target` varchar(100) default '',
  `id_module_group` int(4) unsigned default '0',
  `flag` tinyint(3) unsigned default '0',
  `store_all_data` bool default '0',
  PRIMARY KEY  (`id_agente_modulo`)
) TYPE=InnoDB;
\end{verbatim}

\end{document}
